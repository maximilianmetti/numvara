\documentclass[final,10pt]{article}

\textheight=9.in \textwidth=6.5in \oddsidemargin=0in
\evensidemargin=0in \hoffset=0.cm \voffset=-0.7in

\usepackage{amsmath,graphicx,latexsym,amssymb, amscd, psfrag,verbatim}%,hyperref}
\usepackage{amsfonts,bm,mathrsfs}
\usepackage{algorithmic,algorithm}
\usepackage{color}

\newtheorem{proposition}{Proposition}
\newtheorem{problem}{Problem}
\newtheorem{theorem}{Theorem}
\newtheorem{lemma}{Lemma}
\newtheorem{example}{Example}
\newtheorem{remark}{Remark}
\newtheorem{corollary}{Corollary}

\begin{document}

\title{Energetic Variational Analysis: Notes on Numerics}
\author{Maximilian S.~Metti}
\maketitle


The Energetic Variational Approach is a framework for describing dynamic physical processes.
Most generally, it is rooted in the Laws of Thermodynamics,
the Least Action Principle, the Maximum Dissipation Principle, and Newton's Force Balance Law.
In addition to these basic principles, further modeling assumptions must be made to define a physical system;
namely, an Energy functional and a Dissipation functional must be prescribed,
which are functionals of an initial condition, a flow map, a velocity vector, and a deformation tensor.
In addition, the space of admissible flow maps must be given along with its tangent spaces.
From these given data, calculus of variations are employed to determine the evolution of the dynamical system.


\section{Kinematics and Notation}

Let $\hat{\Omega} \subset \mathbb{R}^d$ and $\Omega(t) \subset \mathbb{R}^d$ for $0 \le t \le T$,
where the volume $\Omega(t)$ deforms continuously with respect to $t$.
Specifically, we assume the existence of a bijective and differentiable \emph{flow map}
that satisfies $\bm{x}(\cdot,t): \hat{\Omega}\rightarrow\Omega(t)$, for each $t$.
When $t=0$, we require $\hat{\Omega} = \Omega(0)$.
From this perspective, $\bm x(\cdot,t)$ is a mapping between volumes.
Alternatively, for a fixed $\hat{\bm x}$ in $\hat{\Omega}$, the flow map $\bm x(\hat{\bm x}, t)$ traces out a trajectory in $\mathbb{R}^{d+1}$.
Let $\bm v(\bm x,t)$ denote the velocity of the flow map satisfying $\bm v(\bm x(\hat{\bm x},t),t) \equiv  \frac{\partial}{{\partial t}}\bm x(\hat{\bm x},t)$;
then, the flow map defines the \emph{flow lines} of the ordinary differential equation,
\[
	\left\{ \begin{matrix}	\bm x_t (\hat{\bm x},t)	&= \bm v \big(\bm x(\hat{\bm x},t),t \big),	&&	t>0	\\
					\bm x (\hat{\bm x},0)	&= \hat{\bm x}.	\end{matrix} \right.
\]
We define the pull-back of the velocity (whenever possible) in the usual way: $\hat{ \bm v}(\hat{\bm x},t) \equiv \bm v(\bm x(\hat{ \bm x},t),t)$.
We define the deformation tensor to be $\hat{\bm F}(\hat{\bm x},t) \equiv \frac{\partial}{{\partial \hat{\bm x}}}\bm x(\hat{\bm x},t)$.
The push-forward of the deformation tensor is given by $\bm F(\bm x(\hat{\bm x},t),t) = \hat{\bm F}(\hat{\bm x},t)$.

For functions defined on the moving frame, $f = f(\bm x,t)$, we define the characteristic (material) derivative as
\[
	\frac{Df(\bm x, t)}{Dt} \equiv \frac{\partial}{\partial t} f \big( \bm x( \bm{\hat{x}}, t), t \big)
			= f_t(\bm x,t) + \bm v(\bm x,t) \cdot \nabla_{\bm x} f(\bm x,t),
\]
where $\bm{\hat{x}} = \bm x^{-1}(\bm x, t) \in \hat{\Omega}$.


\section{Analysis of Mechanics}

The energetic variational approach is a framework connecting energy laws to partial differential equations.
The principles and laws in this approach originate from thermodynamics and classical mechanics.
We begin by introducing the first two Laws of Thermodynamics.

\subsection{Laws of Thermodynamics}

In this section, functionals are denoted using script letters.
Let $\mathcal K(\bm x,\bm v,t)$ represent the {kinetic energy} of a system, 
\[
	\mathcal K(\bm x,\bm v, t) = \int_{\Omega(t)} \frac12 \rho(\bm x, t) \bm v^T(\bm x,t)\bm v(\bm x,t) \, d\bm x,
\]
$\mathcal F(\bm x,\bm F, t)$ the {internal energy}, $\mathcal Q$ the {heat}, and $\mathcal W$ the external work.
The First Law of Thermodynamics states that the rate of change in the kinetic and internal energy is the difference of heat and work:
\begin{equation} \label{first law of thermodynamics}
	\frac{d}{dt}( \mathcal K+\mathcal F)	=	\frac{d\mathcal Q}{dt} + \frac{d\mathcal W}{dt}.
\end{equation}


The Second Law of Thermodynamics relates the temperature $T$, which we take to be a constant, the {entropy} $\mathcal S$, the heat, and the system's \emph{dissipation}, 
$\Delta(\Omega,\bm v, t) \ge 0$,
\begin{equation} \label{second law of thermodynamics}
	T\frac{d}{dt} \mathcal S	=	\frac{d\mathcal Q}{dt} + \Delta.
\end{equation}
This Second Law states that the rate of entropy is determined by dissipation and heat, and can be derived by statistical mechanics
with the assumption of a {fundamental postulate} that ``the system is uncorrelated to some point in the past.''
This postulate asserts that there exists some time in the (distant) past that is far enough from the current time so that the past configuration
of the system at that time has no influence upon the current system.

Taking the difference of (\ref{first law of thermodynamics}) and (\ref{second law of thermodynamics}), 
we relate the \emph{total energy}, $\mathcal H = \mathcal K + (\mathcal F-T\mathcal S) = \mathcal K + \mathcal U$,
where $\mathcal U$ is the potential energy (Helmholtz free energy), to the dissipation,
\[
	\frac{d}{dt}\mathcal H		=	-\Delta + \frac{d\mathcal W}{dt}.
\]
For our purposes, it suffices to consider a \emph{closed} system, where there is no external work present:
\begin{equation}	\label{energy dissipation}
	\frac{d}{dt}\mathcal H		=	-\Delta.
\end{equation}
Since $\Delta \ge 0$, equation (\ref{energy dissipation}) is a natural energy estimate for the mechanical system.

For purposes of variational analysis, it is important to realize that the quantities introduced in this section are integral functionals:
namely, we have
\begin{align*}
	\mathcal H(\bm x, \bm v, \bm F, t)		&=\mathcal K(\bm x, \bm v, t) + \mathcal U(\bm x, \bm F, t)
									= \int_{\Omega(t)} \frac12 \rho(\bm x, t) \bm v^T(\bm x,t)\bm v(\bm x,t) +  U(\bm x, \bm F,t) \, d\bm x\\
\intertext{and}
	\Delta(\bm x, \bm v, t)		&= \int_{\Omega(t)} \gamma(\bm x, t) \Phi(\bm v, \bm v) \, d\bm x.
\end{align*}
We note that $\Delta$ is required to be a symmetric, quadratic, and convex functional of $\bm v$.
The symmetry follows from the assumption of \emph{microscopic reversibility},
which is the assumption that the irreversible macroscopic mechanics are determined reversible microscopic motions and that
any microscopic motion that occurs is only equally as likely to occur as its reverse motion; this assumption constrains the statistical mechanics of the microscopic system. 
The quadratic form follows from \emph{linear response theory}, which requires dissipative forces to be linearly proportional to the velocity.
The convexity of $\Delta$ follows by non-negativity, which is necessary for the second law of thermodynamics to hold.

\subsection{Flow map and velocity spaces}
In many systems, the spaces of the flow map and velocity must be constrained to preserve some physical quality,
such as the conservation of mass or incompressibility.
Thus, we cannot always assume these spaces are linear.
We denote some prescribed space of flow maps for $\bm x \in \mathcal M$ with tangent space $T_{\bm x}\mathcal M$,
and for the velocity $\bm v \in \mathcal M_V$ with tangent space $T_{\bm v}\mathcal M_V$. 



\subsection{Force Balance}
The laws of thermodynamics provide global information about the dynamic system, though they do not give information
on how the system is configured (i.e.~how the density is distributed in space) at any given time.
The configuration of the system can be described by the flow map $\bm x(\bm \hat{x},t)$ and its velocity $\bm v(x,t)$.
In terms of these functions, we define various \emph{forces} present in the system,
\[
	\bm f_i \equiv \bm f_i(\bm x,\bm v,t),	\qquad i=1,\ldots,N_f,
\]
where the relationship between the forces, the flow map, and the velocity must be prescribed in some manner.

To determine the flow map, we require the prescribed forces to satisfy the Law of Force Balance,
which provides a (nonlinear) equation that the flow map and velocity must satisfy.

\begin{proposition} [Force Balance Law]
For closed dynamical systems, the sum of all forces is 0:
\[
	\sum_{i=1}^{N_f} \bm f_i(\bm x,\bm v,t) = \bm{0}.
\]
\end{proposition}

Now, we set up a framework for defining the present forces in a given system,
which requires only expressions for the total energy, the dissipation, and the admissible sets of flow maps and velocities,
along with the tangent spaces for these maps.

In most applications, we require only three forces: the inertial, the conservative, and the dissipative force,
so that our force balance law requires
\[
	\bm f_\mathrm{inert} + \bm f_\mathrm{cons} + \bm f_\mathrm{diss} = 0.
\]
These forces are defined by
\begin{align}
	\bm f_\mathrm{inert}(\bm x, \bm v, t)	&\equiv	-\rho(\bm x, t) \bm v(\bm x, t),	\\
	\bm f_\mathrm{cons}(\bm x, t)		&\equiv	-\frac{\delta \mathcal{U}}{\delta \bm x},	\\
	\bm f_\mathrm{diss}(\bm x, \bm v, t)	&\equiv	-\frac{\delta \frac{_1}{^2} \Delta}{\delta \bm v}.
\end{align}
These forces are defined by the Least Action Principle and the Maximum Dissipation Principle,
which we now show in the remainder of this section.




\subsubsection{Conservative Systems and the Least Action Principle}
The \emph{action} functional is defined as the Legendre transform of the total energy with respect to the velocity, integrated over time: this gives
\[
	\mathcal A(\bm x, \bm v, t)		\equiv	\int_0^T \mathcal K(\bm x, \bm v, t) - \mathcal U(\bm x, t) \, dt.
\]
The Least Action Principle states that the variation of the action is zero for conservative systems.
This implies that the action of a system is at a critical value when the dynamics are driven by kinetic and conservative forces alone.
\begin{proposition}[Least Action Principle]
Suppose the dynamics of a conservative system are described by the flow map and velocity pair, $\bm x$ and $\bm v$.
For a conservative system, the Least Action Principle holds: for all $\bm \delta \bm x \in T_{\bm x}\mathcal M$,
\[
	\delta \mathcal{A}[\bm \delta \bm x(t)]	=	0.
\]
\end{proposition}

Taking the variation with respect to $\bm x$ and an application of integration by parts immediately shows that the Least Action Principle implies
\begin{align*}	\label{least action}
	0 	&= 	\frac{d}{d\varepsilon}\bigg|_{\varepsilon=0} \int_0^T \mathcal K(\bm x_\varepsilon,\bm{v}_\varepsilon,t) 
																	- \mathcal U(\bm x_\varepsilon,t)\, dt	\\
%		&=	\int_0^T \int_{\Omega(t)}	{\rho}({\bm x},t){\bm v}({\bm x}, t)\cdot \frac{_{D}}{^{D t}} \bm \delta \bm x
%										-\frac{\delta \mathcal{U}}{\delta \bm x} \cdot \bm \delta \bm x \, d\bm x dt	\\
		&=	\int_0^T \int_{\Omega(t)}	\bigg( -{\rho}({\bm x},t)\frac{D \bm v}{D t}({\bm x}, t) -\frac{\delta \mathcal{U}}{\delta \bm x} \bigg) \cdot \bm \delta \bm x \, d\bm x dt	\\
		&=	\int_0^T \int_{\Omega(t)} \bigg( \bm f_\mathrm{inert}(\bm x, \bm v, t) + \bm f_{\mathrm{cons}}(\bm x, t) \bigg) \cdot \bm \delta \bm x \, d\bm x dt,
\end{align*}
where $\bm v_\varepsilon = \partial \bm x_{\varepsilon} / \partial t$ and $\bm \delta \bm x = \frac{\partial}{\partial \varepsilon} \big|_{\varepsilon=0}\bm x_\varepsilon$.
We see that in conservative systems, when $f_\mathrm{diss} = 0$, the least action principle is consistent with the law of force balance.



\subsubsection{Dissipative Systems and the Maximum Dissipation Principle}
The \emph{maximum dissipation principal} defines the relationship between the dissipative force and the dissipation functional.
\begin{proposition}\label{Maximimum Dissipation Principle}
The dissipative forces are in the direction of steepest descent of the dissipation funcitonal.
Consequently, the velocity of the dissipation must satisfy
\begin{equation} \label{diss principle}
	\frac{\delta \frac{_1}{^2}\Delta}{\delta \bm v} = -f_\mathrm{diss},
\end{equation}
where the variation is taken with respect to the velocity.
\end{proposition}














\subsection{An example}
\paragraph{Inviscid fluid flow}
Whenever taking the variation with respect to the flow map, it is necessary to write the total energy as a functional in the Lagrangean coordinates.
Thus, we change the coordinates of our energies so that the variation of the Action can be evaluated.
Consider the energy of an inviscid fluid:
\begin{align*}
	\mathcal{H} 	&= \int_{\Omega(t)} \frac12\rho(x,t) \big|\bm v^T(x,t)\big|^2 + \omega( \rho(x,t) )\, dx	\\
				&= \int_{\Omega(0)} \frac12\hat{\rho}(\hat{x},0) \big|\hat{\bm v}(\hat{x},t) \big|^2
								+ \omega\bigg( \frac{\hat{\rho}(\hat{x},0)}{\hat{J}(\hat{x},t)} \bigg){\hat{J}(\hat{x},t)}\, d\hat{x}.
\end{align*}
Thus, the Lagrangean for this system is given by
\[	\mathcal{L}	= \int_{\Omega(0)} \frac12\hat{\rho}(\hat{x},0) \big|\hat{\bm v}(\hat{x},t) \big|^2
								- \omega\bigg( \frac{\hat{\rho}(\hat{x},0)}{\hat{J}(\hat{x},t)} \bigg){\hat{J}(\hat{x},t)}\, d\hat{x}.
\]
Taking the variation of the action, we have by the Least Action Principle
\begin{align*}
	\delta\mathcal A	&= \frac{d}{d\varepsilon}\bigg|_{\varepsilon=0} 
						\int_0^T \int_{\Omega(0)} \frac12\hat{\rho}(\hat{x},0) \big|\hat{\bm v}(\hat{x},t)+\varepsilon\bm\delta\hat{\bm v}(\hat{x},t))\big|^2\\
					&\qquad\quad		- \omega\bigg( \frac{\hat{\rho}(\hat{x},0)}{\det\big(\hat{\bm F}(\hat{x},t)+\varepsilon \bm\delta\bm\hat{F}(\hat{x},t)\big)} \bigg)
								{\det\big(\hat{\bm F}(\hat{x},t)+\varepsilon \bm\delta\bm\hat{F}(\hat{x},t)\big)}\, d\hat{x}dt	\\
					&= \int_0^T \int_{\Omega(0)} \hat{\rho}(\hat{x},0) \hat{\bm v}(\hat{x},t)\cdot\bm\delta\hat{\bm v}
							- \frac{d}{d\varepsilon}\bigg|_{\varepsilon=0} \bigg[
								\omega\bigg( \frac{\hat{\rho}(\hat{x},0)}{\det\big(\hat{\bm F}(\hat{x},t)+\varepsilon \bm\delta\bm\hat{F}(\hat{x},t)\big)} \bigg)
								{\det\big(\hat{\bm F}(\hat{x},t)+\varepsilon \bm\delta\bm\hat{F}(\hat{x},t)\big)}\bigg]\, d\hat{x}dt	\\
				&= \int_0^T \int_{\Omega(0)} -\hat{\rho}(\hat{x},0) \frac{\partial}{\partial t}\hat{\bm v}(\hat{x},t)\cdot\bm\delta\hat{\bm x}
							- \frac{d}{d\varepsilon}\bigg|_{\varepsilon=0} \bigg[
								\omega\bigg( \frac{\hat{\rho}(\hat{x},0)}{\det\big(\hat{\bm F}(\hat{x},t)+\varepsilon \bm\delta\bm\hat{F}(\hat{x},t)\big)} \bigg)
								{\det\big(\hat{\bm F}(\hat{x},t)+\varepsilon \bm\delta\bm\hat{F}(\hat{x},t)\big)}\bigg]\, d\hat{x}dt	\\
					&= \int_0^T \int_{\Omega(0)} -\hat{\rho}(\hat{x},0) \frac{\partial}{\partial t}\hat{\bm v}(\hat{x},t)\cdot\bm\delta\hat{\bm x}
							- \bigg[ \omega_\rho\bigg( \frac{\hat{\rho}(\hat{x},0)}{\hat{J}(\hat{x},t)} \bigg) \bigg(\frac{-\hat{\rho}(\hat{x},0)}{\hat{J}^2(\hat{x},t)}\bigg)
								\big(\nabla_{x} \cdot \bm\delta\hat{\bm x}\big)\hat{J}^2
								+ \omega\bigg( \frac{\hat{\rho}(\hat{x},0)}{\hat{J}(\hat{x},t)}\bigg)\big(\nabla_{x} \cdot \bm\delta\hat{\bm x}\big)\hat{J} \bigg] \, d\hat{x}dt,
\end{align*}
where we used
\[
	\frac{_d}{^{d\varepsilon}}\big|_{\varepsilon=0} \det( \hat{\bm F} + \varepsilon\bm\delta\hat{\bm F} ) 
			= (\nabla_{\bm x} \cdot \bm \delta \hat{\bm x}) \det( \hat{\bm F} + \varepsilon\bm\delta\hat{\bm F} ) \big|_{\varepsilon=0}
			= (\nabla_{\bm x} \cdot \bm \delta \hat{\bm x}) \hat{J}.
\]
We integrate by parts to move the divergence off of the variation $\bm\delta\hat{\bm x}$ to get
\begin{align}
	\delta \mathcal A &= \int_0^T \int_{\Omega(0)} -\hat{\rho}(\hat{x},0) \frac{\partial}{\partial t}\hat{\bm v}(\hat{x},t)\cdot\bm\delta\hat{\bm x}
							- \bigg[ \omega_\rho\bigg( \frac{\hat{\rho}(\hat{x},0)}{\hat{J}(\hat{x},t)} \bigg) \bigg(\frac{-\hat{\rho}(\hat{x},t)}{\hat{J}^2(\hat{x},t)}\bigg)
								\big(\nabla_{x} \cdot \bm\delta\hat{\bm x}\big)\hat{J}^2
								+ \omega\bigg( \frac{\hat{\rho}(\hat{x},0)}{\hat{J}(\hat{x},t)}\bigg)\big(\nabla_{x} \cdot \bm\delta\hat{\bm x}\big)\hat{J} \bigg] \, d\hat{x}dt
								\nonumber\\
					&= \int_0^T \int_{\Omega(t)} -{\rho}({x},t) \frac{D\bm v}{D t}({x},t)\cdot\bm\delta{\bm x}
							- \bigg[ -\rho(x,t) \omega_\rho({\rho}({x},t))\big(\nabla_{x} \cdot \bm\delta{\bm x}\big)
								+ \omega( \rho({x},t) )\big(\nabla_{x} \cdot \bm\delta{\bm x}\big) \bigg] \, d{x}dt\nonumber\\
					&= -\int_0^T \int_{\Omega(t)} \bigg[ {\rho}({x},t) \frac{D\bm v}{D t}({x},t)
							+ \nabla_{\bm x} \Big( \rho(x,t)\omega_\rho({\rho}({x},t))
								- \omega( \rho({x},t) ) \Big) \bigg]\cdot \bm\delta{\bm x} \, d{x}dt \label{inviscid action}\\
					&= 0.
\end{align}
Taking $p = \rho(x,t)\omega_\rho({\rho}({x},t)) - \omega(\rho({x},t))$, we get $f_\mathrm{cons} = -\nabla p$ and
\begin{align}
	\rho_t + \nabla \cdot(\rho v)		&= 0,	\tag{Conservation of Mass}\\
						{\rho} \frac{D\bm v}{D t} + \nabla p	&= \bm 0,	\tag{Least Action Principle}\\
												p	&=\rho\omega_\rho({\rho}) - \omega(\rho).
															\tag{Equation of State}
\end{align}
which corresponds to the inviscid Navier-Stokes.

\paragraph{Incompressibility}
For incompressibility, we must also impose a divergence-free condition on the velocity: $\nabla \cdot \bm v = 0$ and on the variation test functions.
Namely, it is required that
\[
	\det( \hat{\bm F} + \varepsilon \bm\delta\hat{\bm F} ) = 1
\]
so that the variations are volume preserving, which leads to a divergence-free condition condition on $\bm\delta\hat{\bm x}$:
\[
	\nabla\cdot\bm\delta\hat{\bm x} = \frac{d}{d\varepsilon}\bigg|_{\varepsilon=0} \det( \hat{\bm F} + \varepsilon\bm\delta\hat{\bm F} ) = 0.
\]
As in (\ref{inviscid action}), we have
\begin{align*}
	\delta \mathcal A &= \int_0^T \int_{\Omega(0)} -\hat{\rho}(\hat{x},0) \frac{\partial}{\partial t}\hat{\bm v}(\hat{x},t)\cdot\bm\delta\hat{\bm x}
							- \bigg[ \omega_\rho\bigg( \frac{\hat{\rho}(\hat{x},0)}{\hat{J}(\hat{x},t)} \bigg) \bigg(\frac{-\hat{\rho}(\hat{x},0)}{\hat{J}^2(\hat{x},t)}\bigg)
								\big(\nabla_{x} \cdot \bm\delta\hat{\bm x}\big)\hat{J}^2
								+ \omega\bigg( \frac{\hat{\rho}(\hat{x},0)}{\hat{J}(\hat{x},t)}\bigg)\big(\nabla_{x} \cdot \bm\delta\hat{\bm x}\big)\hat{J} \bigg] \, d\hat{x}dt
								\nonumber\\
					&= \int_0^T \int_{\Omega(t)} -{\rho}({x},t) \frac{D\bm v}{D t}({x},t)\cdot\bm\delta{\bm x}
							- \bigg[ -\rho(x,t) \omega_\rho({\rho}({x},t))\big(\nabla_{x} \cdot \bm\delta{\bm x}\big)
								+ \omega( \rho({x},t) )\big(\nabla_{x} \cdot \bm\delta{\bm x}\big) \bigg] \, d{x}dt	\\
					&= \int_0^T \int_{\Omega(t)} -{\rho}({x},t) \frac{D\bm v}{D t}({x},t)\cdot\bm\delta{\bm x} \, d{x}dt,
\end{align*}
though this only holds for divergence-free test functions.
Accordingly, we cannot assume 
\[
	{\rho}({x},t) \frac{D\bm v}{D t}({x},t)	= 0.
\]
Instead, we invoke the \emph{Helmholtz's Decomposition theorem}, which we introduce:
\begin{lemma} [Helmholtz's Decomposition]
We can decompose any smooth vector field into a conservative and divergence-free term:
\[
	\bm w(\bm x) = \nabla p(\bm x) + \big( \nabla\times\bm q + \nabla h \big),
\]
where $h$ is a harmonic function that satisfies $\nabla \cdot \nabla h\equiv0$.
It then follows that if $\int_{\Omega} \bm w \cdot \bm\delta\bm x\,d\bm x=0$ for all divergence free $\bm\delta\bm x$, then $\bm w$ is conservative:
\begin{equation}	\label{conservative field}
	\bm w = \nabla p.
\end{equation}
\end{lemma}

We assume the Helmholtz Decomposition theorem to be given and we show that (\ref{conservative field}) must hold.
Set $\bm\delta\bm x = \nabla\times\bm q + \nabla h.$
Then, we show that $\nabla\times\bm q + \nabla h = \bm 0$:
\begin{align*}
	0	&=	\int_{\Omega} \bm w \cdot \bm\delta\bm x\, d\bm x	\\
		&=	\int_{\Omega} (\nabla p + \nabla\times \bm q + \nabla h) \cdot (\nabla\times\bm q + \nabla h)\, d\bm x	\\
		&=	\int_{\Omega}\nabla p\cdot(\nabla\times \bm q + \nabla h) + |\nabla\times\bm q + \nabla h|^2\, d\bm x	\\
		&=	\int_{\Omega}\nabla\times\nabla p \cdot \bm q - p \nabla\cdot\nabla h + |\nabla\times\bm q + \nabla h|^2\, d\bm x	\\
		&=	\int_{\Omega} |\nabla\times\bm q + \nabla h|^2\, d\bm x,
\end{align*}
as desired.

Hence, for our example of incompressible flow, we have
\[
	{\rho}({x},t) \frac{D\bm v}{D t}({x},t) = -\nabla p_1,
\]
for some differentiable function $p_1$.
We remark that this $p_1$ is a Lagrange multiplier for the incompressibility condition, without physical interpretation.
Thus, we recover the incompressible Navier-Stokes equation:
\begin{align}
	\rho_t + \nabla \cdot(\rho v)			&= 0,	\tag{Conservation of Mass}\\
	{\rho} \frac{D\bm v}{D t} + \nabla {p_1}	&= \bm 0,	\tag{Least Action Principle}\\
					\mathrm{div}\, \bm v	&= 0.	\tag{Incompressibility}
%								p	&=\rho\omega_\rho({\rho}) - \omega(\rho). \tag{Equation of State}
\end{align}








\paragraph{Viscous Fluid Flow}
Consider the example of inviscid fluid flow, as above, though we now introduce a nontrivial dissipation functional corresponding to a viscous forces:
\[
	\Delta(\bm v, \bm v)	=	\int_{\Omega(t)} \frac{\mu}{2} \big( | \nabla_{\bm x} \bm v|^2 + | \nabla_{\bm x}\cdot \bm v|^2 \big) \, dx,
\]
which we note is convex, symmetric, and quadratic in $\bm v$.
We take a variation $\bm\delta\bm v$ that satisfies $\bm \delta\bm v\big|_{\partial\Omega}=0$;
this gives
\begin{align*}
	\delta_{\bm v} \Delta	&=	\frac{d}{d\varepsilon}\bigg|_{\varepsilon=0} 
					\int_{\Omega(t)} \frac{\mu}{2} \big( | \nabla_{\bm x} \bm v + \varepsilon\nabla_{\bm x} \bm\delta\bm v|^2 
									+ |\nabla_{\bm x} \cdot\bm v + \varepsilon\nabla_{\bm x} \cdot\bm\delta\bm v|^2 \big) \, dx\\
				&=	\int_{\Omega(t)} \mu \big( \nabla_{\bm x} \bm v:\nabla_{\bm x} \bm\delta\bm v
									+ (\nabla_{\bm x} \cdot\bm v )(\nabla_{\bm x} \cdot\bm\delta\bm v) \big) \, dx\\
				&=	-\int_{\Omega(t)} \mu \big( \mathrm{div}(\nabla \bm v) + \nabla(\mathrm{div}\,\bm v) \big) \cdot \bm\delta\bm v\, d\bm x\\
				&=	\int_{\Omega(t)} 2f_\mathrm{diss}\cdot \bm\delta\bm v\, d\bm x.
\end{align*}

Thus, we have
\begin{align}
	\rho_t + \nabla \cdot(\rho v)		&= 0,	\tag{Conservation of Mass}\\
	{\rho} \frac{D\bm v}{D t} + \nabla p	&= \mu \frac{\mathrm{div}(\nabla \bm v) + \nabla(\mathrm{div}\,\bm v)}{2},	\tag{Force Balance}\\
							p	&=\rho\omega_\rho({\rho}) - \omega(\rho).
															\tag{Equation of State}
\end{align}
which corresponds to Navier-Stokes for a visous compressible flow.

\paragraph{Incompressibility}
Again, we consider the effects of incompressible flow.
As in the inviscid case, we have
\[
	{\rho}({x},t) \frac{D\bm v}{D t}({x},t)	= \nabla p_1.
\]
For the dissipative forces, we see that $\mathrm{div} \, \bm v = 0$ gives
\[
	\Delta(\bm v, \bm v)	=	\int_{\Omega(t)} \frac{\mu}{2} | \nabla_{\bm x} \bm v|^2\, dx,
\]
leading to
\begin{align*}
	\delta \Delta	&=	\frac{d}{d\varepsilon}\bigg|_{\varepsilon=0} 
					\int_{\Omega(t)} \frac{\mu}{2} | \nabla_{\bm x} \bm v + \varepsilon\nabla_{\bm x} \bm\delta\bm v|^2 \, dx\\
				&=	\int_{\Omega(t)} \mu \nabla_{\bm x} \bm v:\nabla_{\bm x} \bm\delta\bm v \, dx\\
				&=	-\int_{\Omega(t)} \mu\, \mathrm{div}(\nabla \bm v)\cdot \bm\delta\bm v\, d\bm x,\\
				&=	\int_{\Omega(t)} 2 f_\mathrm{diss} \, d\bm x.
\end{align*}
As for the conservative terms, this does not imply $2f_\mathrm{diss}-\mu \mathrm{div}(\nabla \bm v)=0$.
Instead, we use the Helmholtz's Decomposition to show that we have a conservative field,
\[
	\frac{\mu}{2}\mathrm{div}(\nabla \bm v) - f_\mathrm{diss}	= \nabla p_2.
\]
Set $\tilde{{p}} = p_2-p_1$.
Then, the equations for viscous incompressible flow are
\begin{align}
			\rho_t + \nabla \cdot(\rho v)		&= 0,	\tag{Conservation of Mass}\\
	{\rho} \frac{D\bm v}{D t} + \nabla {\tilde{p}}	&= \frac{\mu}{2} \mathrm{div}(\nabla \bm v),	\tag{Force Balance}\\
						\mathrm{div}\, \bm v	&= 0.	\tag{Incompressibility}
\end{align}
























\section{Numerics}


We now turn our attention to the discretization of the force balance law, where we consider the inertial and conservative forces of the Least Action Principle and
the dissipative force of the Maximum Dissipation Principle.
In general, we need to discretize the following problem: find $\bm x \in \mathcal M$ and $\bm v \in \mathcal M_V$ such that
\[
	\bm f_\mathrm{inert} + \bm f_\mathrm{cons} + \bm f_\mathrm{diss} = 0,
\]
where the forces are defined by
\begin{align*}
	\bm f_\mathrm{inert}	&= -\rho \bm v,	\\
	\bm f_\mathrm{cons}	&= -\frac{\delta \mathcal{U}}{\delta \bm x},	\\
	\bm f_\mathrm{diss}	&= -\frac{\delta \frac{_1}{^2} \Delta}{\delta \bm v}.
\end{align*}
In order to discretize, we write the problem in its weak form, though this is not a trivial task since
the dissipative force is obtained by a variation of the velocity, whereas the other forces come from a variation of the flow map.
To work around this difficulty, we consider the semi-discrete problem.


We discretize (\ref{continuous minimization principle}) in a straightforward way;
we partition the time domain
\[
	0 = t_0 < t_1 < \cdots < t_m =T
\]
and define $\Delta t_i = t_i-t_{i-1}$.
This yields a semi-discrete formulation where a collocation scheme discretizes the time variable.
For this manuscript, we restrict our attention to a first order implicit time integration scheme.

Let $\bm x^i(\cdot) = \bm x( \cdot, t_i)$ for $i=0,\ldots,m$ denote the flow map at the time steps.
Whenever ${\bm v}(t_i) = \frac{\bm x_i - \bm x_{i-1}}{\Delta t_i}$, we can write the dissipative force as
\[
	\bm f_\mathrm{diss}(\bm v) 	= -\frac{\delta \frac{_1}{^2} \Delta}{\delta \bm v}(\bm v) 
							= -\Delta t_i \frac{\delta \frac{_1}{^2} \Delta}{\delta \bm x_i}\left(\frac{_{\bm x_i - \bm x_{i-1}}}{^{\Delta t_i}}\right),
\]
which allows us to recast the problem in weak form: for $i=1,\ldots, m$, find $\bm x_i \in \mathcal M$ such that
\[
	\left( -\rho(\hat{\bm x}) \frac{\bm x_i - \bm x_{i-1}}{\Delta t_i} -\frac{\delta \mathcal{U}}{\delta \bm x_i}\big(\bm x_i\big)
			-\Delta t_i \frac{\delta \frac{_1}{^2} \Delta}{\delta \bm x_i}\left(\frac{_{\bm x_i - \bm x_{i-1}}}{^{\Delta t_i}}\right), \bm \delta\bm x\right) =0
\]
for all $\bm \delta \bm x \in \mathcal M$.
We recognize that for each time step, this is equivalent to the minimization problem
\[
	\min_{\bm x_i \in \mathcal M} \int_{\hat{\Omega}} \frac12\hat{\rho}\frac{|\bm x_i - \bm x_{i-1}|^2}{\Delta t_i^2} + U(\bm x_i)
				+ \frac{\Delta t_i}{2}\gamma \Phi\left(\frac{{\bm x_i - \bm x_{i-1}}}{{\Delta t_i}},\frac{{\bm x_i - \bm x_{i-1}}}{{\Delta t_i}}\right)\, d\bm x,
\]
where we note $\hat{\rho} = \hat{\rho}(\hat{\bm x})$ and $\gamma = \gamma(\hat{\bm x})$.



To fully discretize the problem, we want to approximate each $\bm x^i_h \approx \bm x^i \in \mathcal M$.
We define a finite dimensional subspace $\mathcal{M}_h \subset \mathcal M$ for the flow maps, $\bm x_h^i$.
To construct $\mathcal M_h$, we fix some triangulation initial $\hat{\mathcal T}_h$ of the fixed domain, $\hat \Omega$,
and define $\mathcal A (\hat{\mathcal T}_h)$ as the set of affine equivalent triangulations to $\hat{\mathcal T}_h$.
Then, the set $\mathcal M_h$ is given by
\[
	\mathcal{M}_h \equiv \Big\{ \bm x \, \big| \, \bm x: \hat{\mathcal{T}}_h \mapsto \mathcal T_h' \in \mathcal A(\hat{\mathcal T}_h) \Big\}.
\]
Essentially, each $\bm x_h \in \mathcal M_h$ corresponds to a family of affine isoparametric maps from elements in $\hat{\mathcal T}_h$
to elements in some triangulation in $\mathcal A (\hat{\mathcal{T}}_h)$.
We then define the discretized velocity as $\bm v_h(t_i) \equiv \frac{\bm x^i - \bm x^{i-1}}{\Delta t_i}$.

This gives the discrete problem












These maps take triangulations of the domain $\Omega(0)$ to (potentially curvilinear) triangulations of some new domain $\Omega(t)$
and can be defined in terms of the location of the nodes of the triangulations;
let $\{ \bm x^n_i \}_{i=0}^N$ denote the collection of nodes in the triangulation at time $t=t_n$, then
\[
	\bm x^n_h(\hat{\bm x}) = \bm x^n_h( \bm x^n_0,\ldots,\bm x_N^n; \hat{\bm x}).
\]


We note that the mechanics underlying the systems considered above are summarized by the force-balance equation of proposition \ref{force balance}.
\[
	\frac{\delta (\frac{_1}{^2}\Delta)}{\delta \bm v}-\frac{\delta \mathcal{A}}{\delta \bm x}	= \bm 0.
\]
Let $\mathcal{M}$ denote the set of all injective differentiable maps $\bm x:\Omega\times[0,T]\rightarrow\mathbb{R}^d$ satisfying $\bm x(\hat{\bm x},t) = \hat{\bm x} \in \Omega(0)$
and $\det \partial \bm x/ \partial \hat{\bm x} > 0$.
We show that the force balance equation is a necessary condition for the minimizer of the following optimization problem.

\begin{lemma}
Let $\bm x(\cdot,t):\Omega(0)\rightarrow\Omega(t)$ be the flow map minimizing
\begin{equation}	\label{continuous minimization principle}
	\lim_{\Delta t \rightarrow 0^+} \min_{\bm x^*(t)\in \mathcal M}
					\ \left[ \frac{_{\Delta t}}{^2} \Delta\bigg(\frac{\bm x^*(t)-\bm x(t-\Delta t)}{\Delta t},\frac{\bm x^*(t)-\bm x(t-\Delta t)}{\Delta t}\bigg)
					- \mathcal{A}\big( \bm x^*(t) \big) \right]
\end{equation}
for all $0 < t \le T$ also satisfies the force balance equation.
\end{lemma}


Recall that the dissipation takes the form:
\[
	\Delta(\bm v(t), \bm v(t)) 	= \int_{\Omega(t)} {\rho}(\bm x)\Phi\big(\bm v(\bm x,t),\bm v(\bm x,t) \big) \, d\bm x.
\]
We use a simple consequence of the chain rule:
\[
	\frac{\partial}{\partial {\bm v}(t)} \Phi\big({\bm v}({\bm x},t),{\bm v}({\bm x},t) \big) 
		=	\lim_{\Delta t\rightarrow0^+}\Delta t \frac{\partial}{\partial {\bm x}(t)} 
			\Phi\bigg(\frac{\bm x(t)-\bm x(t-\Delta t)}{\Delta t},\frac{\bm x(t)-\bm x(t-\Delta t)}{\Delta t} \bigg),
\]
to show for all $\bm \delta \bm y$ that
\[
	\frac{\delta}{\delta \bm v(t)}\frac{_1}{^2}\Delta(\bm v, \bm v)[\bm \delta \bm y] 
		= \lim_{\Delta t \rightarrow 0^+} \left\{ \Delta t \frac{\delta}{\delta \bm x(t)}\frac{_1}{^2}
			\Delta\bigg(\frac{\bm x(t)-\bm x(t-\Delta t)}{\Delta t},\frac{\bm x(t)-\bm x(t-\Delta t)}{\Delta t}\bigg)[\bm \delta \bm y] \right\}.
\]
Thus, a necessary condition for the minimizer of (\ref{continuous minimization principle}) is
\begin{align*}
	0	&=\lim_{\Delta t \rightarrow 0^+} \left\{ \Delta t \frac{\delta}{\delta \bm x(t)}\frac{_1}{^2}
				\Delta\bigg(\frac{\bm x(t)-\bm x(t-\Delta t)}{\Delta t},\frac{\bm x(t)-\bm x(t-\Delta t)}{\Delta t}\bigg)
			-	\frac{\delta}{\delta \bm x(t)} \mathcal{A}(\bm x(t))	\right\}	\\
		&=	\frac{\delta\frac{_1}{^2}\Delta}{\delta \bm v(t)}(\bm v(t), \bm v(t)) -	\frac{\delta\mathcal A}{\delta \bm x(t)}(\bm x(t)).
\end{align*}
		

We discretize (\ref{continuous minimization principle}) in a straightforward way;
we partition the time domain
\[
	0 = t_0 < t_1 < \cdots < t_m =T
\]
and define $\Delta t_i = t_i-t_{i-1}$.
This yields a semi-discrete formulation where a collocation scheme discretizes the time variable.
Let $\bm x^n(\cdot) = \bm x( \cdot, t_n)$ for $i=0,\ldots,m$.

We define a finite dimensional subspace $\mathcal{M}_h \subset \mathcal M_h$ of bounded injective maps $\bm x_h$ with domain $\Omega(0)$.
For our purposes, we assume that these maps are given by a finite collection of isoparametric maps, as used in finite element methods.
These maps take triangulations of the domain $\Omega(0)$ to (potentially curvilinear) triangulations of some new domain $\Omega(t)$
and can be defined in terms of the location of the nodes of the triangulations;
let $\{ \bm x^n_i \}_{i=0}^N$ denote the collection of nodes in the triangulation at time $t=t_n$, then
\[
	\bm x^n_h(\hat{\bm x}) = \bm x^n_h( \bm x^n_0,\ldots,\bm x_N^n; \hat{\bm x}).
\]


From this perspective, we minimize a discretization of (\ref{continuous minimization principle}) by rearranging the nodes in the triangulation,
as in moving mesh methods.
Thus, the discrete problem we solve is:
\begin{equation}	\label{discrete problem}
	\min_{\bm x_h^* \in \mathcal{M}_h} \left\{ \frac{_{\Delta t_n}}{^2}\Delta \bigg( \frac{\bm x_h^*-\bm x_h^{n-1}}{\Delta t_n},\frac{\bm x_h^*-\bm x_h^{n-1}}{\Delta t_n}\bigg)
			- \mathcal{A}(\bm x_h^*,\bm F_h) \right\}.
\end{equation}
The minimizer is taken to be the discrete flow map $\bm x_h^n$.
To guarantee the existence of such a minimizer, we require convexity of the dissipation and the total energy.
It can be helpful to use Hermite isoparametric maps so that the Lagrangean density $\hat{\rho}_h(\hat{\bm x},t) = \hat{\rho}(\hat{x},0)/\det(\hat{\bm F}_h(\hat{x},t))$
is continuous.

\paragraph{Adaptivity}
In most applications, the action functional tends to infinity if $\det \partial \bm x/ \partial\hat{\bm x} \rightarrow 0$ anywhere on the domain;
this corresponds to the great amount of energy required to truly compress a control volume into a surface, curve, or point.
Nevertheless, spatial adaptivity can improve the convergence rates of the optimization problem as line searches in the numerical optimization process
have greater flexibility in the magnitude of the mesh updates.
This is true when the flow map is compressing in a particular region, and a coarser triangulation can be used as it compresses without such a strong influence on the energy.
Furthermore, algorithms for time adaptivity can often be formulated as decreasing the total energy by some prescribed amount at each time step.
These techniques fit quite naturally into this problem, where we can either control the dissipation or the total energy.




\subsection{Test Problems}

We consider a few simple test problems of a single spatial dimension with $\Omega(0) = [-1,1]$ to demonstrate this method.
Although the functions in the the discrete optimization problem (\ref{discrete problem}) are convex, it is nontrivial to find the minimizer.
Attempts to minimize using the conjugate gradient method and quasi-Newton Methods (e.g.~with BFGS approximate Hessians) would
not converge, even imposing the Wolfe conditions for the line search of the update stepsize.
In the case of quasi-Newton with BFGS, the Wolfe conditions can be used to prove non-singularity of the approximate Hessian,
though this matrix became numerically singular.
Thus, we use Newton's method (with the analytic Hessian) to solve the minimization problem.
Since the functions are convex, the Hessian is positive definite.
We present the result of some experiments below.

\subsubsection{Heat Equation}

Let $\hat{\rho}(\cdot)$ be the given initial density function.
The energy of the heat equation is given by
\[
	\mathcal H (\bm x,   F) = \int_{\Omega(t)} \rho(  x,t) \log\left({\rho(  x,t)}\right)\, d  x
	=	\int_{\Omega(0)} \hat{\rho}(\hat{  x}) \log \big(\hat{\rho}(\hat{  x})/ \det\hat{  F}(\hat{  x},t)\big) \, d\hat{  x},
\]
where we used the conservation of mass $\rho(  x(\hat{  x},t),t) =  \hat{\rho}(\hat{  x})/\det\hat{  F}(\hat{  x},t)$.
(Note that there is no kinetic energy.)
The dissipation is given by Darcy's damping,
\[
	\Delta (  v,   v)	=	\int_{\Omega(t)} \rho(  x,t) \big|  v(  x,t) \big|^2\, d  x	
					=	\int_{-1}^1 \hat{\rho}(\hat{  x}) \big|\hat{  v}(\hat{  x},t) \big|^2\, d\hat{  x}.
\]
Following the examples in \S 3, one can show this leads to the heat equation.

In the case of a single spatial dimension, the flow maps take the form
\[
	  x_h^n(\hat{  x}) = \sum_{i=0}^N x_i^n b_i(\hat{  x}),
\]
where the basis functions $b_i$ are the nodal basis functions for a finite element space.
Furthermore, 
\[
	\det\hat{F}_h(\hat{ x},t_n) = {x}_i^n - {x}^n_{i-1} \equiv \Delta \hat{x}_i,	\quad	\hat{ x}_{i-1} \le \hat{ x} \le \hat{ x}_i.
\]

Hence, the function we aim to minimize is
\begin{multline*}
	\int_{-1}^1 \frac{\Delta t}{2}\hat{\rho}(\hat{  x}) \left(\frac{x_h^n(\hat{  x}) - x_h^{n-1}(\hat{  x})}{\Delta t}\right)^2
							+ \hat{\rho}(\hat{  x}) \log\left(\frac{\hat{\rho}(\hat{  x})}{\det\hat{F}_h(\hat{  x},t)}\right) \, d\hat{  x}\\
			=	\sum_{i=1}^N \Delta \hat{x}_i \int_0^1 
						\hat{\rho}(\hat{x}) \frac{\big[\hat{x}(x_i^n-x_i^{n-1})-(1-\hat{x})(x_{i-1}^n-x_{i-1}^{n-1})\big]^2}{2\Delta t}
							+ \hat{\rho}(\hat{  x}) \log\left(\frac{\hat{\rho}(\hat{  x})}{\Delta x_i^n}\right) \, d\hat{x}.
\end{multline*}


\paragraph{References}
\begin{itemize}
	\item M. Chen, C. Liu, S. Majd, S. Xu, ``Modeling and Simulating Asymmetrical Conductance Changes in Gramicidin Pores.'' 
	\item R. Eisenberg, Y. Hyon, C. Liu, ``Energy Variational Analysis of Ions in Water and Channels: Field Theory for Primitive Models of Complex Ionic Fluids.''
			J. Chemical Phys. 133,104104 (2010).
	\item S.V. Fomin, I.M. Gelfand, ``Calculus of Variations.'' Prentice-Hall, Inc.  Englewood Cliffs, New Jersey (1963).
	\item Johannes Forster, ``Methematical Modeling of Complex Fluids'' Master's Thesis, University of W\"urzberg (2013).
	\item C. Liu, ``An Introduction of Elastic Complex Fluids: An Energetic Variational Approach.''  Dept.~Math., Penn State University.
	\item L. Onsager, ``Reciprocal Relations in Irreversible Processes. I'' Physical Review 31, pp.~405--426 (1931).
	\item L. Onsager, ``Reciprocal Relations in Irreversible Processes. II'' Physical Review 31, pp.~2265--2279 (1931).
\end{itemize}

\end{document}